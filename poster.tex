\documentclass[portrait]{a0poster}
% aus a0poster die interessanten Teile kopieren und forken
\RequirePackage{environ}

\NewEnviron{Feld}[2]{%
\ptextbox{#1}{#2}{\BODY}%
}

\usepackage[utf8]{inputenc}
\usepackage{microtype}
\usepackage{wrapfig}
\usepackage{amsmath}
\usepackage{amsfonts}
\usepackage{anyfontsize}

\usepackage{tikz}
\usetikzlibrary{positioning}

\definecolor{background}{rgb}{0.78,0.80,0.90}
\definecolor{bcgsred}{rgb}{0.71,0.18,0.18}
\definecolor{bcgsblue}{rgb}{0.09,0.19,0.35}
\definecolor{plotblue}{rgb}{0.18,0.18,0.71}
\definecolor{plotgreen}{rgb}{0.18,0.71,0.18}

\pagecolor{background}

\tikzstyle{bcgsbox} = [draw=bcgsblue, fill=white, line width=3pt,
rounded corners=1cm, inner sep=1.5cm, inner ysep=2cm]

\tikzstyle{bcgsreset} = [draw=black, fill=none, thin, sharp corners,
inner sep=0.3333em, node distance=1cm and 1cm]

\tikzstyle{bib} = [inner sep=0.1cm, inner ysep=0.1cm]

\newcommand{\MyTitle}{Electromagnetic switching of multiferroics}
\newcommand{\MyAuthors}{M. Baum$^1$, J. Stein$^1$, Th. Finger$^1$, N. Qureshi$^1$, J. Leist$^3$, S. Holbein$^1$, P. Becker-Bohatý$^2$, L. Bohatý$^2$, G. Eckold$^3$, M. Braden$^1$}
\newcommand{\MyAffiliations}{$^1$ II. Physikalisches Institut, Universität zu Köln\\
$^2$ Institut für Kristallographie, Universität zu Köln\\
$^3$ Institut für Physikalische Chemie, Universität Göttingen}


\renewcommand{\familydefault}{\sfdefault}

\makeatletter %% crucial for the following to work! ...
\renewcommand{\section}{\@startsection
  {section}                          % the name
  {1}                                % the level
  {0mm}                              % the indent
  {-0\baselineskip}                  % the beforeskip
  {.5\baselineskip}                 % the afterskip
  {\normalfont\LARGE\bfseries\color{bcgsred}}} % the style
\makeatother

\makeatletter %% crucial for the following to work! ...
\renewcommand{\subsection}{\@startsection
  {subsection}                          % the name
  {2}                                % the level
  {0mm}                              % the indent
  {0.6\baselineskip}                  % the beforeskip
  {0.25\baselineskip}     % the afterskip
  {\normalfont\Large\bfseries\color{bcgsblue}}} % the style
\makeatother

\renewcommand{\normalsize}{\large}

\newcommand{\ptextbox}[3]{%
  \node[bcgsbox, #2] (#1){%
    \begin{minipage}{0.42\textwidth}%
      #3
    \end{minipage}
  };
}

\newcommand{\ptextboxbib}[3]{%
  \node[bib, #2] (#1){%
    \begin{minipage}{0.45\textwidth}%
      #3
    \end{minipage}
  };
}

\addtolength{\leftmargini}{4mm} % first-level list leftmargin
\addtolength{\labelsep}{4mm}
\addtolength{\leftmarginii}{6mm} % second-level list leftmargin



%%% Local Variables: 
%%% mode: latex
%%% TeX-master: "poster"
%%% End: 


\usepackage[a0paper,top=10cm, left=1cm, right=1cm]{geometry}


\begin{document}

\iffalse
\begin{tikzpicture}[node distance=2cm and 1cm]

\node[bcgsbox, draw=white, fill=bcgsblue] (title){%
    \begin{minipage}{0.9\textwidth}
      \parbox[t]{0.9\textwidth}{%
      \parbox[t]{9cm}{\vspace{0pt}   \includegraphics[height=8cm]{unikoeln-logo2}}
    \hspace{1cm}
    \parbox[t]{50cm}{%
        \textcolor{white}{\raggedright\Huge\textbf{\MyTitle}
          \Large \\[1ex]
          \MyAuthors\\
          {\small \MyAffiliations \\}}}
}
     \parbox[t]{8cm}{\vspace{0pt} \includegraphics[height=8cm]{Logo-SFB-608}}
    \end{minipage}
};

\end{tikzpicture}
\fi


\begin{tikzpicture}[node distance=2cm and 1cm]
\node[bcgsbox, draw=white, fill=bcgsblue] (title){%
    \begin{minipage}{0.9\textwidth}
\parbox[t]{0.2\textwidth}{\includegraphics[height=8cm]{unikoeln-logo2}}
\parbox[t]{0.5\textwidth}{Titeltext}
\parbox[t]{0.2\textwidth}{\hfill\includegraphics[height=8cm]{unikoeln-logo2}}
\end{minipage}
};

\begin{Feld}{motivation}{below left=of title.south}
  \section{Motivation}
  \begin{itemize}
  \item bla
  \end{itemize}
\end{Feld}

\ptextbox{boxtwo}{below=of motivation}{
  \section{Magnetic structure}
  \subsection{subtitle}
  \begin{equation*}
    1 + 2 = 3
  \end{equation*}
  As given in~\cite{Gen99}.
}

\ptextbox{results}{below right=of title.south}{
  \section{Results}
\subsection{subtitle}
}

\ptextbox{outlook}{below=of results}{
\section{Conclusion}
\begin{itemize}
\item bla
\end{itemize}
}

\ptextboxbib{boxbib}{below=of boxtwo}{
  \large
  \begin{thebibliography}{M}
  \bibitem{Gen99} \textbf{Implausible Results} A.~Genius, Journal of
    Research \textbf{99} 99 (2099) 999999
  \end{thebibliography}
 Poster für Nobelpreisverleihung 2020 vom \today.
}

\end{tikzpicture}
\end{document}
